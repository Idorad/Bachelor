\documentclass[12pt]{beamer}
\usetheme{Berlin}
\usepackage[utf8]{inputenc}
\usepackage[german]{babel}
\usepackage[T1]{fontenc}
\usepackage{amsmath}
\usepackage{amsfonts}
\usepackage{amssymb}
\author{Busch \& Tiersch}
\title{Beobachter Muster}
%\setbeamercovered{transparent} 
%\setbeamertemplate{navigation symbols}{} 
%\logo{} 
%\institute{} 
%\date{} 
%\subject{}
\begin{document}

\begin{frame}
\titlepage
\end{frame}

\begin{frame}{Gliederung}
\tableofcontents
\end{frame}
\section{Was}
\begin{frame}{Was}
  Gehört zu den Entwurfsmustern.
  Vorteil:
  \begin{itemize}
  \item Zeitersparnis durch die Wiederverwendung von bewährten Mustern
  \item Fehlerfreiheit
  \item Gemeinsame Kommunikationsgrundlage
  \item Sauberes OO-Design
  \end{itemize}    
\end{frame}
\section{Wozu}
\begin{frame}{Wozu}
  Grob gesagt damit ein Programm auf bestimmte Ereignisse reagieren soll:
  \begin{itemize}
    \item Änderung von Werten
    \item Aktualisierung im Allgemeinen
    \item Daten darstellen und bei einem Update die Veränderung sofort übernehmen.
  \end{itemize}
\end{frame}
\begin{frame}{Wozu}
  \begin{enumerate}
    \item jedes Objekt hat aktuellen Zustand.
    \item von Änderungen sind andere Objekte abhängig 
    \item anhängige Objekte müssen benachrichtigt werden
    \item abhängigen Objekte sind Observer
    \item zu beobachtende Objekt sind Subject
  \end{enumerate}
\end{frame}
\section{Arbeitsweise}
\begin{frame}{Arbeitsweise}

\end{frame}
\section{Arbeitsweise in UML}
\begin{frame}{Arbeitsweise in UML}

\end{frame}
\section{Beispielanwendung}
\begin{frame}{Beispielanwendung}

\end{frame}
\subsection{Beispielanwendung klasse1}
\begin{frame}{Beispielanwendung klasse1}

\end{frame}
\subsection{Beispielanwendung klasse2}
\begin{frame}{Beispielanwendung klasse2}

\end{frame}
\subsection{Beispielanwendung header1}
\begin{frame}{Beispielanwendung klasse1}

\end{frame}
\subsection{Beispielanwendung header2}
\begin{frame}{Beispielanwendung klasse2}

\end{frame}

\end{document}